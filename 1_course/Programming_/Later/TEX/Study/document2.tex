\documentclass[a4paper,12pt]{report}
\usepackage[T2A]{fontenc}
\usepackage[utf8]{inputenc}
\usepackage[english,russian]{babel}
\usepackage{circuitikz}
\usepackage{wrapfig}
\usepackage{makecell}
\usepackage{tabularx}
\usepackage{graphicx}
\usepackage{gensymb}
\usepackage{cancel} %cancel symbol
\usepackage{amsmath,amsfonts,amssymb,amsthm,mathtools}

%tikz (draw)

\usepackage{tikz}

%tikz libraries

\usetikzlibrary{intersections}
\usetikzlibrary{arrows.meta}
\usetikzlibrary{calc,angles,positioning}

\usepackage{float}

\parindent=0ex

\graphicspath{ {C:/Users/George/Documents/MIPT_TEX/} }

\newcommand{\R}{{\mathbb R}}
\newcommand{\N}{{\mathbb N}}
\newcommand{\fancy}[1]{{\mathbb{#1}}}
\DeclareMathOperator{\sgn}{sgn}
\newtheorem{problem}{Задача}[chapter]
\newenvironment{sol}{\paragraph{Решение}}{}
\renewcommand\thesection{\arabic{section}}
\newcommand{\uni}{\cup}
\newcommand{\inter}{\cap}

\begin{document}
	\begin{titlepage}
	\begin{center}
		МОСКОВСКИЙ ГОСУДАРСТВЕННЫЙ УНИВЕРСИТЕТ \\
		ИМ. М.В. ЛОМОНОСОВА \\
		
		
		\hfill \break
		Факультет вычислительной математики и кибернетики\\
		\vspace{2.5cm}
		\large{\textbf{Отчет по заданию № 1}}\\
		\hfill \break
		\\
	\end{center}
	
	\begin{flushright}
		Автор:\\
		Студент гр. 106\\
		Кондрашов Д.С.
	\end{flushright}
	
	\vspace{7cm}
	
	\begin{center}
		\includegraphics[width=0.3\linewidth]{msu_logo}
	\end{center}
	
	
	
	
	\vfill
	
	\begin{center} Москва, 2024 \end{center}
	
	\thispagestyle{empty}
	
\end{titlepage}
	\newpage
	\pagenumbering{arabic}
	
	\tableofcontents
	
	\chapter{Алгебра, математический анализ}
	\addcontentsline{toc}{chapter}{Алгебра, математический анализ}
	
	\begin{problem}
		Найти уравнение прямой, проходящей через точку $A(2;1)$ и перпендикулярной прямой $y=x+1$
	\end{problem}
	\begin{sol}
		Пусть искомая прямая имеет уравнение $y=kx+b$. Запишем условия перпендикулярности и принадлежности точки $A$:
		\[
		\begin{cases}
			k\cdot 1 = -1\\
			1=k\cdot2 + b
		\end{cases}
		\]
		Решая систему получаем $k=-1$ и $b=3$. Искомая прямая имеет уравнение: \fbox{$y=-x+3$}
		\qed
	\end{sol}
	
	\begin{problem}
		Построить графики функций:
		\[
		a) \hspace{2mm} y=\cos x - \sin x \hspace{5mm} b) \hspace{2mm} y=3\cos(2x+\pi / 4) \hspace{5mm} c) \hspace{2mm} y=2|x+1|+3
		\]
	\end{problem}
	\begin{sol}
		Нарисовать график читатель сможет сам, используя графический калькулятор. Однако тут покажем алгоритм действий, чтобы построить подобные графики:\\
		a) Для начала необходимо воспользоваться тождеством:
		\[
		\cos x - \sin x = \sqrt{2}\cos({x+\pi/4})
		\]
		Затем с помощью цепочки элементарных преобразований из графика $y=\cos x$ строим искомый график следующим образом:
		\begin{enumerate}
			\item Строим $y=\cos x$
			\item Сдвигаем на $\pi/4$ влево по оси $x$
			\item Растягиваем по оси $y$ в $\sqrt{2}$ раз.
		\end{enumerate}
		b) Строится похожим образом, что и a), однако тут есть уловка. Может показаться, что график получается стягиванием в 2 раза по $x$ и смещением на $\pi/4$ влево, однако это не так. В действительности можно сделать просто эти два действия наоборот, но важно понимать, что сдвиг на самом деле будет на $\pi/8$, а не на $\pi/4$. Это аргументируется следующим образом:
		\[
		3\cos (2x+\pi/4)=3\cos (2(x+\pi/8))
		\]
		Этот график уже получается растягиванием в 3 раза по $y$, стягиванием в 2 раза по $x$ и смещением на $\pi/8$.\\
		c) График получается элементарными преобразованиями с $y=|x|$:
		\begin{enumerate}
			\item Строим $y=|x|$
			\item Сдвигаем на 1 влево по $x$
			\item Растягиваем в два раза по $y$
			\item Смещаем на 3 вверх по $y$
		\end{enumerate}
		\qed
	\end{sol}
	
	\begin{problem}
		Найти область определения функции $y=\log_{x+0.5}(2x^2-7x+6)$
	\end{problem}
	
	\begin{sol}
		Для решения такой задачи необходимо знать ограничения для логарифма (или другой функции, которая определена не везде). Пусть дана функция
		\[
		y=\log_{f(x)}g(x)
		\]
		Ограничения в таком случае будут следующие:
		\begin{equation*}
			\begin{cases}
				f(x)>0\\
				f(x)\neq1\\
				g(x)>0
			\end{cases}
		\end{equation*}
		Используя эту систему находим ограничения для нашей функции:\\
		\vspace{-5mm}
		\begin{center}
		\fbox{$D_y=(-0.5,0.5)\uni(0.5,1.5)\uni(2,+\infty)$}
		\end{center}
		\qed
	\end{sol}
	
	\begin{problem}
		Найдите все значения параметра \(p\), при которых корни квадратного трехчлена $3x^2+6x+p-1$ различны и удовлетворяют неравенству $x_1^2+x_2^2\leq5$
	\end{problem}
	\begin{sol}
		Решать будем с помощью формулы Виета. Давайте их сначала выведем, потому что это лучше, чем запоминать.\\
		Пусть квадратный трехчлен раскладывается следующим образом: 
		\[
		ax^2+bx+c=a(x-x_1)(x-x_2)
		\]
		где $x_1$ и $x_2$ -- корни. Тогда раскрываем скобки:
		\[
		ax^2-x\cdot a(x_1+x_2)+ax_1x_2
		\]
		Откуда получаем
		\[
		\begin{cases}
			x_1+x_2=-b/a\\
			x_1x_2=c/a
		\end{cases}
		\]
		Как же получить $x_1^2+x_2^2$? Все просто, возводим $x_1+x_2$ в квадрат и получаем:
		\[
		(x_1+x_2)^2=x_1^2+x_2^2+2x_1x_2
		\]
		\[
		x_1^2+x_2^2=b^2/a^2-2c/a
		\]
		Подставляя $a$, $b$, $c$ из исходного выражения получаем
		\[
		36/9-2(p-1)/3\leq5
		\]
		Откуда находим $p$:
		\begin{center}
			\fbox{$p\geq-0.5$}
		\end{center}
		\qed
	\end{sol}
	
	\begin{problem}
		\label{Extremum}
		Найти наибольшее значение функции $f(x)=(1-x^2-2x)^3$
	\end{problem}
	\begin{sol}
		Решать будем с помощью производной.
		\[f'(x)=3(1-x^2-2x)^2(-2x-2)\]
		Находить промежутки монотонности функции будем с помощью решений уравнения $f'(x)=0$ и метода интервалов.\\
		Заметим, что множитель $(1-x^2-2x)^2$ положителен или равен нулю. Для начала упростим $f'(x)$:
		\[
		f'(x)=-6(x^2+2x-1)^2(x+1)
		\]
		Можно выделить полный квадрат:
		\[
		f'(x)=-6((x+1)^2-2)^2(x+1)
		\]
		\[
		f'(x)=-6(x+1+\sqrt{2})^2(x+1-\sqrt{2})^2(x+1)
		\]
		\begin{figure}[H]
			\centering
			\begin{tikzpicture}
				%\draw [help lines] (0,0) grid (8,2);
				%\draw[color=black!60, very thick](4,4) circle(4);
				\draw [arrows = {-Latex[width=0pt 5, length=5pt]},very thick] (1,1) -- (7,1) node [pos = 1, above = 1.5mm] {$f'(x)$};
				%\node at (7,1.5) (f') {$f'(x)$};
				\draw [thin] (2,0.8) -- (2,1.2);
				\draw [thin] (3.5,0.8) -- (3.5,1.2);
				\draw [thin] (5,0.8) -- (5,1.2);
				\node at (1.5,1.3) (-1) {$+$};
				\node at (2.75,1.3) (-2) {$+$};
				\node at (4.25,1.3) (+1) {$-$};
				\node at (5.75,1.3) (+2) {$-$};
				\node [scale=0.9] at (2,0.5) (a1) {$-1-\sqrt{2}$};
				\node [scale=0.9] at (3.5,0.45) (a2) {$-1$};
				\node [scale=0.9] at (5,0.5) (a3) {$-1+\sqrt{2}$};
			\end{tikzpicture}
		\end{figure}
		Очевидно максимум будет в точке $-1$ (однако стоит это все-таки аргументировать). Нас просят найти значение функции в точке максимума, поэтому придется посчитать. Подставляя получим:
		\begin{center}
			\fbox{$\max f(x)=f(-1)=(1-1+2)^3=8$}
		\end{center}
		\qed
	\end{sol}
	\begin{problem}
		Решить неравенство:
		\[
		\frac{x^2+|x|-12}{x-3}\geq2x
		\]
	\end{problem}
	\begin{sol}
		Решать такие неравенства можно разными способами. Иногда от модуля можно полностью избавиться, сделав замену, однако в данном случае это невозможно. Всегда можно раскрыть модуль по определению. Тогда получится два неравенства.
		\begin{equation*}
			\begin{cases}
				x\neq3\\
				\left[
				\begin{array}{ll}
					x^2+x-12\geq2x^2-6x, & \hspace{2mm} x\geq0\\
					x^2-x-12\geq2x^2-6x, & \hspace{2mm} x < 0
				\end{array}
				\right .
			\end{cases}
		\end{equation*}
		Решая систему получим ответ:
		\begin{center}
			\fbox{$(-\infty,3)\uni(3,4]$}
		\end{center}
		\qed
	\end{sol}
	\begin{problem}
		Решите уравнение: $x^4+2x^3-11x^2+4x+4=0$
	\end{problem}
	\begin{sol}
		К сожалению такие уравнения решаются подбором. Надо хорошо знать схему Горнера и уметь быстро (и желательно без ошибок) считать. Иногда есть некоторые хитрости, которые можно использовать в свою пользу. Часто в таких задачах первый корень находится очень просто, обычно это $\pm1$. Хорошо бы знать симметричные и иные типы уравнений, которые решаются своими способами.\\
		
		В данном случае схема Горнера и решение квадратного уравнения дают решения:
		\begin{center}
			\fbox{$x\in\{1,2,\frac{-5\pm\sqrt{17}}{2}\}$}
		\end{center}
		\qed
	\end{sol}
	\begin{problem}
		Является ли функция $y=\cos^3x-5\sin x + x^2$ четной или нечетной.
	\end{problem}
	\begin{sol}
		Для решения необходимо понимать, что происходит с четностью/нечетностью функции, когда мы берем от нее другую функцию. Для этого необходимо уметь пользоваться теоремами:\\
		
		Легко проверить, что композиция четных функций четна, композиция нечетных -- нечетна. Также легко проверить, что четная от нечетной -- нечетна, нечетная от четной -- четна. Это нам дает все инструменты для решения подобных задач.\\
		
		Куб косинуса -- четен, синус -- нечетен, $x^2$ -- четен. Сумма таких функций есть функция \fbox{общего вида}
		\qed
	\end{sol}
	
	\begin{problem}
		Найти период функции $y=\cos{x}\cos{2x}-\sin{x}\sin{2x}$
	\end{problem}
	\begin{sol}
		Давайте для начала вспомним формулы перехода от произведения к сумме. А точнее вспомним как их вывести:\\
		\begin{eqnarray*}
			\cos{(x+y)}=\cos{x}\cos{y}-\sin{x}\sin{y}\\
			\cos{(x-y)}=\cos{x}\cos{y}+\sin{x}\sin{y}
		\end{eqnarray*}
		Отсюда можно найти:
		\begin{eqnarray*}
			\cos{x}\cos{y}=\frac{1}{2}(\cos{(x+y)}+\cos{(x-y)})\\
			\sin{x}\sin{y}=\frac{1}{2}(\cos{(x-y)}-\cos{(x+y)})
		\end{eqnarray*}
		Подставим в исходную функцию:
		\[
		y=\frac{1}{2}(\cos{3x}+\cos{(-x)})-\frac{1}{2}(\cos{(-x)}-\cos{3x})
		\]
		Пользуясь четностью косинуса приходим к упрощенному виду:
		\[
		y=\cos{3x}
		\]
		Отсюда период, очевидно, \fbox{$T=2\pi/3$}
		\qed
	\end{sol}
	
	\begin{problem}
		Найти точки разрыва функции и определить их вид. Найти асимптоты графика:
		\[
		f(x)=\frac{x+1}{(x-2)(x^2+2x-3)}
		\]
	\end{problem}
	\begin{sol}
		Для начала разложим знаменатель на множители:
		\[
		f(x)=\frac{x+1}{(x-2)(x+3)(x-1)}
		\]
		Тут к сожалению все точки разрыва скучные, поэтому давайте немного усложним задачу: пусть наша функция имеет вид
		\[
		f(x)=\frac{x+1}{(x-2)(x-3)(x+1)}
		\]
		Для того, чтобы определить тип точек разрыва, необходимо посчитать пределы функции вокруг них. Если пределы слева-справа равны и конечны, или оба конечны (но не обязательно равны), то это точка разрыва первого рода (в случае равенства пределов такой разрыв называют устранимым). Если же предел хотя бы с одной стороны бесконечен, то разрыв второго рода. Давайте считать.
		\[
		\lim\limits_{x\to-1-0}{f(x)}=\frac{1}{12} \qquad \lim\limits_{x\to-1+0}{f(x)}=\frac{1}{12}
		\]
		Видим, что пределы слева и справа равны и конечны, значит разрыв устранимый (1 рода).
		Проделываем то же самое с другими точками:
		\[
		\lim\limits_{x\to2-0}{f(x)}=+\infty \qquad \lim\limits_{x\to2+0}{f(x)}=-\infty
		\]
		\[
		\lim\limits_{x\to3-0}{f(x)}=-\infty \qquad \lim\limits_{x\to3+0}{f(x)}=+\infty
		\]
		Как видим, точки 2 и 3 -- точки разрыва второго рода.\\
		Теперь найдем асимптоты. Асимптоты бывают наклонные и вертикальные. Наклонные -- прямые, заданные уравнением $y=kx+b$ (в частном случае $k=0$ асимптота называется горизонтальной). Вертикальные -- прямые $x=c$, где $c\in \R$ -- некоторая константа.\\
		Две вертикальные асимптоты \fbox{$x=2$} и \fbox{$x=3$} мы уже нашли, давайте проанализируем на наклонные.\\
		Коэффициент $k$ находится по формуле:
		\[
		k=\lim\limits_{x\to\pm\infty}{\frac{f(x)}{x}}
		\]
		$k$, вообще говоря, существует только когда предел в $+\infty$ равен пределу в $-\infty$.
		$b$ находится по формуле:
		\[
		b=\lim\limits_{x\to\pm\infty}{(f(x)-kx)}
		\]
		$b$ тоже существует, если пределы равны.\\
		Используя эти формулы находим:
		\[
		k=0, \qquad b=0
		\]
		Тогда прямая \fbox{$y=0$} -- горизонтальная асимптота.
		\qed
	\end{sol}
	\begin{problem}
		\label{Extremum2}
		Найти промежутки монотонности и точки экстремума функции $f(x)=x^3+3x^2-2$
	\end{problem}
	\begin{sol}
		Мы уже решали похожую задачу \ref{Extremum} (она даже немного сложнее). Поэтому здесь приведем только вычисление производной и ответ:
		\[
		f'(x)=3x^2-6x
		\]
		\begin{center}
			\fbox{$f(x)$ убывает на $[-2,0]$}\\
			\vspace{1mm}
			\fbox{$f(x)$ возрастает на $(-\infty,-2]; \hspace{2mm} [0,+\infty)$}\\
			\vspace{1mm}
			\fbox{$x=-2$ -- точка максимума, $x=0$ -- точка минимума}
		\end{center}
		Хочется обратить внимание на то, что ответ, где вместо точки с запятой стоит знак $\uni$ -- неправильный. В этом можно убедиться построив график функции и применив определение возрастания функции на этом множестве. Конкретнее, оно говорит, что для \textit{любого} $x_1 < x_2$ выполняется $f(x_1)<f(x_2)$. Если на множестве $(-\infty,-2]\uni[0,+\infty)$ мы возьмем $x_1=-2$ и $x_2=0$, то $x_1<x_2$, но $f(-2)=2>0=f(0)$
		\qed
	\end{sol}
	\begin{problem}
		Исследовать на монотонность $f(x)=x^3-3x^2+5x-4$
	\end{problem}
	\begin{sol}
		Задача идентична \ref{Extremum2}, поэтому решение приведено не будет.
	\end{sol}
	\begin{problem}
		Найти критические точки, исследовать на монотонность $y=x+\cos{x}$
	\end{problem}
	\begin{sol}
		Тут уже задача поинтереснее, давайте также найдем асимптоты графика функции. Для начала найдем производную:
		\[
		f'(x)=1-\sin{x}
		\]
		Производная определена везде, поэтому ищем критические точки, решая уравнение $f'(x)=0$. Оно равносильно
		\[
		\sin{x}=1\ \Leftrightarrow\  x=\frac{\pi}{2}+2\pi k, \quad k \in \fancy{Z}
		\]
		Критических точек -- бесконечное множество. С промежутками монотонности уже интереснее, так как $f'(x)\geq0 \ \forall x\in \R$, поэтому можно сказать, что \fbox{$f(x)$ возрастает на всем $\R$}.\\
		Теперь разберемся с асимптотами:\\
		\fbox{Вертикальных нет}, так как нет точек разрыва. Наклонные ищем по уже известным формулам:
		\[
		k=\lim\limits_{x\to\pm\infty}{\frac{f(x)}{x}}=\lim\limits_{x\to\pm\infty}{\frac{x+\cos{x}}{x}}=\lim\limits_{x\to\pm\infty}\left(1+\frac{\cos{x}}{x}\right)
		\]
		В силу ограниченности косинуса предел равен 1.
		\[
		b=\lim\limits_{x\to\pm\infty}{(x+\cos{x}-x)}=\lim\limits_{x\to\pm\infty}{\cos{x}}
		\]
		Видим, что предела нет. Значит \fbox{ наклонных асимптот тоже нет }.
		\qed
	\end{sol}
	\begin{problem}
		На кривой $y=2x^2-x+15$ найти точку, в которой касательная параллельна прямой $y=-3x+1$
	\end{problem}
	\begin{sol}
		Задача опять на применение производной. Ловушка заключается в том, что касательная может совпасть, это тоже надо бы проверить. Для начала найдем производную:
		\[
		f'(x)=4x-1
		\]
		Уравнение касательной в точке $x=x_0$:
		\[
		y-y_0=f'(x_0)(x-x_0), \quad y_0=f(x_0)
		\]
		Условие параллельности:
		\[
		f'(x_0)=4x_0-1=-3 \Rightarrow x_0=-1/2 \rightarrow f(x)
		\]
		\[
		f(x_0)=f(-1/2)=2\cdot\frac{1}{4}+\frac{1}{2}+15=16
		\]
		Подставляем в уравнение касательной
		\[
		y=-3\left(x+\frac{1}{2}\right)+16\quad y=-3x+\frac{29}{2}
		\]
		Она очевидно не совпала с $y=-3x+1$, значит точка касания, которую нас просили найти:
		\begin{center}
			\fbox{$\left(-1/2;16\right)$}
		\end{center}
		\qed
	\end{sol}
	
	\begin{problem}
		Найдите сумму координат точки пересечения касательной, проведенной к графику функции $f(x)=2x^2-9x-4$ в его точке с абсциссой, равной 3, с осью ординат.
	\end{problem}
	\begin{sol}
		Условие достаточно запутанное, давайте разберемся, что от нас требуют: нам надо провести касательную к графику $f(x)$ в точке $x_0=3$, пересечь это касательную с осью ординат и найти сумму координат этой точки. Раз пересечение с осью ординат, то $x=0$, значит сумма координат равна $y$ (высоте над $x$).\\
		Найдем производную:
		\[
		f'(x)=4x-9
		\]
		Уравнение касательной в $x=x_0=3$
		\[
		y-f(3)=f'(3)(x-3)
		\]
		\[
		y=3(x-3)-13, \quad y=3x-22
		\]
		Пересечение с осью ординат найдем подстановкой $x=0$. Получаем \fbox{-22} \qed
	\end{sol}
	
	\chapter{Геометрия}
	Решать мы будем только задачи, требующие аналитических выкладок, либо задачи на векторы.
	\begin{problem}
		В ДСК найти вектор, ортогональный плоскости (ABC), площадь $\triangle$ABC, если A(1,2,1), B(0,3,-1), C(4,1,2)
	\end{problem}
	\begin{sol}
		Начнем с вектора, ортогонального ABC. Найти его можно с помощью векторного произведения, которое по определению перпендикулярно обоим векторам из него. Для этого найдем координаты векторов AB, AC:
		\[
		AB(-1,1,-2),\quad AC(3,-1,1)
		\]
		На самом деле нам хватит и двух векторов, что хорошо.\\
		
		Найти координаты вектора, перпендикулярного плоскости (его еще называют нормалью к плоскости) можно вычислив определитель матрицы:
		\begin{equation*}
			\begin{vmatrix}
				\vec{i} & \vec{j} & \vec{k}\\
				-1 & 1 & -2\\
				3 & -1 & 1
			\end{vmatrix}=\vec{i}(1-2)-\vec{j}(-1+6)+\vec{k}(1-3)=\fbox{(-1,-5,-2)}
		\end{equation*}
		Если мы найдем модуль этого вектора, то он по определению равен площади \textit{параллелограмма}, построенного на этих векторах. Площадь $\triangle$ABC будет равна половине этого модуля.
		\begin{equation*}
			S\triangle ABC = \frac{1}{2}|[AB,AC]|=\frac{1}{2}\sqrt{1+25+4}=\frac{\sqrt{30}}{2}
		\end{equation*}
		\qed
	\end{sol}
	
	\begin{problem}
		Представить вектор $\vec{d}(-1,1,5)$ в виде линейной комбинации векторов $\vec{a}(3,0,2)$, $\vec{b}(-3,2,4)$, $\vec{c}(1,1,1)$.
	\end{problem}
	\begin{sol}
		Линейной комбинацией векторов называется сумма
		\[
		\alpha_1 \vec{a} + \alpha_2 \vec{b} + \alpha_3 \vec{3}
		\]
		где $\alpha_i\in \R$ -- некоторый скаляр. Пользуясь свойствами координат при умножении на скаляр и суммировании получим, что координаты вектора $\vec{d}$ -- $d_x, \ d_y, \ d_z$ выражаются через координаты исходных векторов следующим образом:
		\begin{equation*}
			\begin{cases}
				d_x=\alpha_1 a_x + \alpha_2 b_x + \alpha_3 c_x\\
				d_y=\alpha_1 a_y + \alpha_2 b_y + \alpha_3 c_y\\
				d_z=\alpha_1 a_z + \alpha_2 b_z + \alpha_3 c_z
			\end{cases}
		\end{equation*}
		Решив эту систему из 3 уравнений с 3 неизвестными $\alpha_1, \ \alpha_2, \ \alpha_3$ получим:
		\begin{center}
			\fbox{$\alpha_1 = 1, \quad \alpha_2 = 1, \quad \alpha_3 = -1$}
		\end{center}
		\qed
	\end{sol}
	
	\begin{problem}
		Выяснить, являются ли 2 вектора $\vec{a}_1(-1,-2,5)$ и $\vec{a}_2(2,-3,1)$ линейно зависимыми. Будут ли линейно зависимы $\vec{a}_1, \ \vec{a}_2, \ \vec{a}_3$, если $\vec{a}_3(1,0,6)$
	\end{problem}
	\begin{sol}
		Тут нужно вспомнить критерий коллинеарности векторов, очевидно, что $\vec{a}_1$ и $\vec{a}_2$ не коллинеарны, а значит линейно независимы.\\
		
		Теперь надо вспомнить когда система из трех векторов является линейно зависимой. Это так в случае, когда векторы компланарны. Значит нужно использовать критерий компланарности. Найдем определитель (ориентированный объем параллелепипеда, построенного на этих векторах)
		\begin{equation*}
			\begin{vmatrix}
				-1 & -2 & 5\\
				2 & -3 & 1\\
				1 & 0 & 6
			\end{vmatrix}=18-2+0-(-15-24+0)=55\neq0
		\end{equation*}
		Векторы некомпланарны, а значит \fbox{линейно независимы}
		\qed
	\end{sol}
	
	\begin{problem}
		Лежат ли точки A(1,2,-1), B(0,1,5), C(-1,2,1), D(1,2,3) в одной плоскости.
	\end{problem}
	\begin{sol}
		Достаточно проверить компланарность векторов AB, AC, AD. Если они компланарны, то и все точки лежат в одной плоскости. Для этого запишем координаты каждого из этих векторов и, аналогично предыдущему, посчитаем определитель.
		\[
		AB(-1,-1,6), \quad AC(-2,0,2), \quad AD(0,0,4)
		\]
		\begin{equation*}
			\begin{vmatrix}
				-1 & -1 & 6\\
				-2 & 0 & 2\\
				0 & 0 & 4
			\end{vmatrix}=4\cdot(-2)=-8 \neq 0
		\end{equation*}
		Векторы некомпланарны, значит \fbox{точки не лежат в одной плоскости}
		\qed
	\end{sol}
	\begin{problem}
		Даны 4 точки A(1,1,-2), B(2,3,7), C(0,4,7), D(-1,2,-2). Докажите, что четырёхугольник ABCD - прямоугольник.
	\end{problem}
	\begin{sol}
		Найдем векторы, которые задают все 4 стороны.
		\[
		AB(1,2,9), \quad BC(-2,1,0), \quad CD(-1,-2,-9),\quad DA(2,-1,0)
		\]
		Достаточно будет показать, что скалярное произведение векторов, задающих соседние стороны, равно нулю. Посчитаем:
		\begin{eqnarray*}
			(AB,BC)=&-2+2+0&=0\\
			(BC,CD)=&2-2+0&=0\\
			(CD,DA)=&-2+2+0&=0\\
			(DA,AB)=&2-2+0&=0
		\end{eqnarray*}
		\qed
	\end{sol}
	
	\chapter{Тригонометрия}
	\begin{problem}
		Вычислить $\sin{(0.5\arccos{(1/9)})}$
	\end{problem}
	\begin{sol}
		Обозначим
		\[
		\alpha = \arccos{(1/9)}
		\]
		Нам нужно найти $\sin{0.5\alpha}$, а мы знаем, что $\cos{\alpha}=1/9$. Для начала надо вспомнить формулу половинного угла. Вытащим мы ее из формул понижения степени
		\[
		\sin^2{\frac{\alpha}{2}}=\frac{1-\cos{\alpha}}{2}
		\]
		Откуда
		\[
		\sin{\frac{\alpha}{2}}=\pm \sqrt{\frac{4}{9}}=\pm\frac{2}{3}
		\]
		\qed
	\end{sol}
	
	\begin{problem}
		Решить уравнение
		\[
		\sin{2x}\cos{(x+\pi/3)}+\sin{(x+\pi/3)}\cos{2x}=\sqrt{3}/2
		\]
	\end{problem}
	\begin{sol}
		Можно просто раскрыть синус и косинус суммы, однако лучше заметить, что перед нами знакомая формула -- синус суммы. Тогда вся левая часть сворачивается
		\[
		\sin{(3x+\pi/3)}=\sqrt{3}/2
		\]
		Решая относительно $3x+\pi/3$ получаем
		\begin{equation*}
			\left[ 
			\begin{array}{ll}
				3x+\pi/3=\pi/3+2\pi n, &\quad n\in \fancy{Z}\\
				3x+\pi/3=2\pi/3+2\pi k, &\quad k\in \fancy{Z}
			\end{array}
			\right.
		\end{equation*}
		\begin{equation*}
			\left[ 
			\begin{array}{ll}
				x=2\pi n/3, &\quad n\in \fancy{Z}\\
				x=\pi/9+2\pi k/3, &\quad k \in \fancy{Z}
			\end{array}
			\right.
		\end{equation*}
		\qed
	\end{sol}
\end{document}